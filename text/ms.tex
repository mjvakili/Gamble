\documentclass[12pt, preprint]{aastex}
\usepackage[breaklinks,colorlinks, urlcolor=blue,citecolor=blue,linkcolor=blue]{hyperref}
\usepackage{graphicx}	% For figures
\usepackage{natbib}	% For citep and citep
\usepackage{amsmath}	% for \iint
\usepackage{mathtools}
\usepackage{bm}
\usepackage[breaklinks]{hyperref}	% for blackboard bold numbers
\usepackage{hyperref}
\usepackage{algorithmic,algorithm}
\hypersetup{colorlinks}
\usepackage{color}
\usepackage{morefloats}
\definecolor{darkred}{rgb}{0.5,0,0}
\definecolor{darkgreen}{rgb}{0,0.5,0}
\definecolor{darkblue}{rgb}{0,0,0.5}
\hypersetup{ colorlinks,
linkcolor=darkblue,
filecolor=darkgreen,
urlcolor=darkred,
citecolor=darkblue}

\DeclareMathOperator*{\argmax}{arg\,max}

\newcommand{\todo}[1]{{\em \textcolor{red}{ #1}}}
\newcommand{\beq}{\begin{equation}}
\newcommand{\eeq}{\end{equation}}
\newcommand{\lang}{\langle}
\newcommand{\ra}{\rangle}
\newcommand{\vep}{\bm{\epsilon}}
\newcommand{\ep}{\epsilon}
\newcommand{\pars}{\vec{\theta}}
\newcommand{\dev}{\mathrm{d}}
\begin{document}

\title{Can we constrain galaxy assembly bias with group and clustering statistics of SDSS galaxies?}
\author{Mohammadjavad Vakili , others}

\begin{abstract}

The standard halo occupation modeling approach in large scale structure 
studies assumes that the halo mass alone suffices in characterizing the 
galaxy-halo connection. Galaxy assembly bias, the hypothesis that properties beyond 
halo mass influences the galaxy-halo connection, can potentially affect the 
statistical summaries of galaxy populations. The degree to which, assembly bias can 
can be constrained with the statistical distribution of galaxies remains largely 
untested. In this investigation, we make use of the recently proposed assembly bias decoration 
of the HOD modeling in order to find constraints on the strength of the assembly bias from the most up-to-date and publicly available measurements of the group multiplicity function and the projected two-point correlation function of the Volume-limited luminosity threshold sample of SDSS 
galaxies. We find that the available GMF measurements cannot constrain galaxy assembly bias in any of the Luminosity thresholds. On the other hand, we find non-zero constraints on the satellite galaxy assembly bias for the $M_{r}<-18$ and $M_{r}<-19$ samples, nonzero constraints on the central galaxy assembly bias for the $M_{r}<-20$ samples, and constraints consistent with zero for the assembly bias of galaxies in the $M_{r}<-21$ sample. Furthermore, we provide a model comparison between the standard HOD approach and the decorated HOD in modeling of the GMF and the projected 2PCF measurements.

\end{abstract}

\section{Introduction}
The assumption that galaxies reside dark matter halos 
is the central tenet of the modern cosmology and 
large-scale structure formation theories. 
One of the most powerful methods for describing 
the galaxy-halo connection is the halo occupation modeling. 
The underlying assumption of the traditional HOD framework is 
that the halo mass alone is sufficient in determining the
population of galaxies inside halos. That is, the statistical properties of 
galaxies is governed by the halo mass. Mathematically, this assumption can be 
written as $P(N_g|M_h,\{x\})=P(N_g|M_h)$ where $\{x\}$ is the set of all 
possible halo properties beyond mass $M_{h}$. 

Despite this very simplifying assumption, the galaxy-halo connection models such as HOD and CLF
have been widely and successfully used in modeling the measurements of a wide range of the 
statistical properties of galaxies such as . However, the assumption that the statistical 
distribution of galaxies can be determined by halo mass is not a derived characteristic of 
accurate and high resolution hydrodynamic simulations, but rather a theoretical assumption 
that needs to be scrutinized. In recent years, the idea that the   

In this investigation, we explore the notion that how the dependence of 
galaxy-halo connection on halo properties beyond mass (e.g. concentration, age).
Throughout this paper, unless stated otherwise, 
all radii and densities are in comoving units. Standard flat $\Lambda$CDM is assumed, 
and all cosmological parameters are set to the Planck 2013 best-fit estimates.

\section{Method}
\subsection{Occupation modeling}
For our standard HOD modeling, HOD parameterization of cite[zheng07] is assumed. According to this parameterization, the occupation of the central galaxies follows a nearest-integer distribution, 
and the occupation of the satellite galaxies follows a Poisson distribution. The expected number of centrals and satellites as a function of the host halo mass of $M_{\rm h}$ are given by the folllowing equations 
\begin{eqnarray}
\langle N_{\rm cen}(M_{\rm h}) \rangle &=& \frac{1}{2}\Big[1+\Big(\frac{\log M_{\rm h} - \log M_{\rm min}}{\sigma_{\log \rm{M}}} \Big) \Big], \label{hod:central}\\ 
\langle N_{\rm sat}(M_{\rm h}) \rangle &=& \Big( \frac{M_{\rm h} - M_{\rm{0}}}{M_{\rm 1}} \Big)^{\alpha}. \label{hod:satellite}
\end{eqnarray}

For populating the halos with galaxies, we follow the procedure described in CITE[HEARIN ETAL], CITE[CHANG ETAL]. The central galaxies are assumed to be at the center of the host dark matter halos. We place the satellite galaxies are within the virial radius of the halo following an NFW profile [CITE NFW]. The concentration of the NFW profile is given by the empirical mass-concentration relation provided by [CITE MACCIO]. The velocities of the satellite galaxies are given by two components. The first component is the velocity of the host halo. The second component is the velocity of the satellite galaxy with respect to the host halo which is computed following the solution to the NFW profile Jeans equations [CITE MORE]. We refer the readers to [CITE HEARIN 15] and \url{http://halotools.readthedocs.io} for a more comprehensive and detailed discussion of the forward modeling of the galaxy mock catalogs. 

Now let us provide a brief overview of HOD modeling decorated with $\mathtt{Heaviside}$ $\mathtt{Assemblybias}$ introduced in [CITE HEARIN2015]. At a fixed halo mass $M_{\rm h}$, halos are split into two populations: population of halos with the 0.5-percentile of highest concentration, and population of halos with 0.5 percentile of lowest concentration.We call the first population "type 1" halos, and the second population "type 2 halos. Following In the decorated HOD model, the expected number of central and satellite galaxies in the two populations is given by

\begin{eqnarray}
\langle N_{c,i} | M_{h},c\rangle &=& \langle N_{c} | M_{h}\rangle + \delta N _{c,i}, \; i=1,2 \label{eq:decoratedcentral} \\
\langle N_{s,i} | M_{h},c\rangle &=& \langle N_{s} | M_{h}\rangle + \delta N _{s,i}, \; i=1,2 \label{eq:decoratedsatellite}
\end{eqnarray}
where $\langle N_{c} | M_{h}\rangle$ and $\langle N_{s} | M_{h}\rangle$ are given by Eqs \ref{hod:central} and \ref{hod:satellite} respectively, and we have $\delta N_{s,0} + \delta N_{s,1} = 0$, and $\delta N_{c,0} + \delta N_{c,1} = 0$. These two conditions ensure the conservation of HOD. At a given host hale mass $M_{h}$, the central occupation of the the two populations follows a Nearest-integer distribution with the first moment given by \ref{eq:decoratedcentral}; and the central occupation of the the two populations follows a Poisson distribution with the first moment given by \ref{eq:decoratedsatellite}. 

In this decorated occupation model, the allowable range for the quantities $\delta N_{c,1}$ and $\delta N_{s,1}$ is given by 
\begin{eqnarray}
-\langle N_{c} | M_{h}\rangle \leq \delta N_{c} \leq \langle N_{c} | M_{h}\rangle
 , \label{eq:cen-bounds} \\
-\langle N_{s} | M_{h}\rangle \leq \delta N_{s} \leq \langle N_{s} | M_{h}\rangle. \label{eq:sat-bounds}
\end{eqnarray}
Afterwards, CITE[HEARIN2015] defines the assembly bias parameter $\mathcal{A}$ in the following way:
\begin{eqnarray}
\delta N_{\alpha , 1}(M_{h}) &=& \mathcal{A_\alpha} \delta N_{\alpha , 1}^{\rm max}(M_{h}) \; \; \rm{if} \; \mathcal{A_\alpha} > 0,  \\
\delta N_{\alpha , 1}(M_{h}) &=& \mathcal{A_\alpha} \delta N_{\alpha , 1}^{\rm min}(M_{h}) \; \; \rm{if} \; \mathcal{A_\alpha} < 0,
\end{eqnarray}
where the subscript $\alpha = c , s$ for the centrals and satellites respectively, and $\delta N_{\alpha , 1}^{\rm max}(M_{h})$, $\delta N_{\alpha , 1}^{\rm min}(M_{h})$ are given by Eqs. \ref{eq:cen-bounds} and \ref{eq:sat-bounds}.  
\subsection{Observables}

\subsection{}
For sampling from the likelihood given, we use the affine-invariant MCMC sampler $\mathtt{emcee}$ (CITE[FORMAN-MACKEY]).


\section{Data}

In this work, we focus on two sets of measurements made on the volume limited luminosity threshold main sample of galaxies 
in the SDSS spectroscopic survey. The first set consists of the number density $n_{g}$ (measured in units of $h^{3}\rm{Mpc}^{-3}$) 
and the projected two point correlation function of galaxies $w_{\rm{p}}(r_{\rm{p}})$ (measured in units of $h^{-1}\rm{Mpc}$), 
and their corresponding covariance matrices constructed using 400 jackknife sub-samples. 
The 2PCFs are measured in 12 logarithmic bins (in units of $h^{-1}\rm{Mpc}$) of width $\Delta \log(r_{\rm p}) = 0.2$, 
starting from $r_{\rm{p}} = 0.1 \; h^{-1}\rm{Mpc}$. $w_{\rm{p}}(r_{\rm{p}})$s are measured by integrating the two dimensional 
2PCF $\xi(r_{\rm{p}} , \pi)$ along the line-of-sight up to $\pi_{max}=40 h^{-1}\rm{Mpc}$. The advantage of using these measurements 
is that the effects of fiber collision systematics on the two-point statistics are corrected for, and therefore, they provide very 
accurate small scale clustering measurements [see CITE GUO 2015].

The Second set of measurements used in our work are the group multiplicity functions described in [CITE BERLIND 2006]. These measurements are made 
for the three Volume limited samples of the SDSS galaxies with luminosity thresholds of $M_{r}<-18$, $M_{r}<-19$, and $M_{r}<-20$. 
Group multiplicity function $g(N)$ (with units of $h^{3}\rm{Mpc}^{-3}$) is defined as the abundance of the galaxy groups as a 
function of group richness (the number of galaxies in groups) per richness width, measured in bins of richness. 
Galaxy groups in [CITE BERLIND etal] are found by Friends-of-Friends (hereafter FoF) group finder algorithm [CITE FOF]. 
The FoF group finder places two galaxies in a group if the projected (line-of-sight) distance between the two galaxies 
is less than a specified projected (line-of-sight) linking length in units of the mean inter-particle distance ($n_{\rm g}^{-1/3}$).
The projected linking and the line-of-sight linking lengths are denoted by $b_{\perp}$ and $b_{\parallel}$ respectively. 
In order to minimize the discrepancy between the groups found from the group finder and the halos in a suite of simulations, 
[CITE BERLIND] set $b_{\perp}$ and $b_{\parallel}$ to 0.15 and 0.74 respectively.
The measurements are accompanied by two sets of uncertainty measurements. 
The first one is the error in $g(N)$ measured from dispersion between $g(N)$s 
of different mock catalogs, and the second source of error is the Poisson error measured 
from the number of groups in each richness bin. Finally, these two uncertainties are added in quadrature 
to estimate the total error on $g(N)$ measurements.

\section{Simulation}
For the simulations used in this work, we make use of the Rockstar (cite[Behroozi]) halo catalogs in the Bolshoi-Planck high 
resolution $N$-body simulation CITE[Klypin2014]. This simulation was carried out using the adaptive refinement tree code 
(ART, see cite[Kravstov]) code, following the Planck $\Lambda$CDM cosmological parameters 
$\Omega_{\rm m} = 0.307$, $\Omega_{\rm b} = 0.048$, $\Omega_{\Lambda} = 0.693$, $\sigma_{8} = 0.823$, $n_{\rm s}=0.96$, 
$h=0.678$. The Box size for this $N$-body simulation is 250 $h^{-1} \rm{Mpc}$, the number of simulation particles is 2048$^3$, 
the mass per simulation particle $m_{\rm p}$ is $1.5 \times 10^{8} \; h^{-1} M_{\odot}$, and the gravitational softening length 
$\epsilon$ is 1 $h^{-1} \rm{kpc}$. 

\section{Results}



\section{Discussion}


\end{document}
