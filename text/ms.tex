\documentclass[12pt, preprint]{aastex}
\usepackage{graphicx}	% For figures
\usepackage{natbib}	% For citep and citep
\usepackage{amsmath}	% for \iint
\usepackage{bm}
\usepackage[breaklinks]{hyperref}	% for blackboard bold numbers
\usepackage{hyperref}
\hypersetup{colorlinks}
\usepackage{color}
\usepackage{morefloats}
\definecolor{darkred}{rgb}{0.5,0,0}
\definecolor{darkgreen}{rgb}{0,0.5,0}
\definecolor{darkblue}{rgb}{0,0,0.5}
\hypersetup{ colorlinks,
linkcolor=darkblue,
filecolor=darkgreen,
urlcolor=darkred,
citecolor=darkblue }

\DeclareMathOperator*{\argmax}{arg\,max}

\newcommand{\beq}{\begin{equation}}
\newcommand{\eeq}{\end{equation}}

\begin{document}

\title{Constraining galaxy assembly bias with galaxy-galaxy measurements of Sloan Digital Sky Survey}

\begin{abstract}

\end{abstract}

\section{Introduction}

Throughout this paper, unless stated otherwise, all radii and densities are in comoving units.
Standard flat $\lambda$CDM is assumed, and all cosmological parameters are set to the planck 
2015 best-fit estimates.

\section{Galaxy assembly bias}

The assumption that galaxies reside dark matter halos is the bedrock of the modern cosmology and 
large-scale structure formation theories. One of the most powerfuls model for descibing the galaxy-halo 
connection is the halo occupation modeling. The underlying assumption of the traditional HOD modeling is 
that the halo mass alone is sufficient in determining the population of halos. Despite its simplifying 
assumption, this model has ben widely used in fitting the measurements of galaxy clustering an galaxy-galaxy lensing.

\section{Galaxy-Galaxy Lensing}

In g-g lensing, the connection between dark matter and galaxies is probed by measuring 
the cross correlation between the two. In particular, one can define the galaxy-matter cross correlation 
as the ensemble average of the product of galaxy over densities and matter overdensities
\beq
\xi_{gm}(r) = \langle \delta_{g}(\vec{x}) \delta_{m}(\vec{x}+\vec{r}) \rangle,
\eeq
where $\delta_g$ and $\delta_m$ are overdensities of matter and galaxies respectively and we have 
assumed the cross-corrlation depends only on the radial separation due to isotropy in distribution of matter
and galaxies.

This cross correlation can be related to the azimuthally-averaged  projected surface density of the gravitational
lenses,
\begin{eqnarray}
\Sigma(R) &=& \bar{\rho}_{m}(z)\int_{0}^{\infty} \;  [1 + \xi_{gm}(\sqrt{\chi^2 + R^2})] \; d\chi \nonumber \\
          &=& 2\bar{\rho}_{m}(z)\int_{R}^{\infty} \; [1 + \xi_{gm}(r)] \; \frac{rdr}{\sqrt{r^{2} - R^{2}}},
\end{eqnarray}
where $\chi$ is the projected comoving line-of-sight distance, and $R$ is the projected comoving transverse distance.

Since the galaxy-matter cross correlation falls off rapidly that in practice thre redshoft evolution of $\xi_{gm}$ 
can be neglected, integrating only out to a line-fo-sight distance of $\chi=$ 50 Mpc is sufficient. 
The mean excess projected density $\bar{\Sigma}(R)$ can be approximated by the following radial integral,
\beq
\bar{\Sigma}(R , z_{l}) = 2 \rho_{c,0}\Omega_{m,0}\int_{0}^{\mathrm{50Mpc}} \; \xi_{gm}\Big(\sqrt{R^{2}+\chi^{2}},z_{L} \Big)d\chi.
\eeq

\section{Concentration-dependence of galaxy-halo connection}

\section{Method}

\section{SDSS measurements}

\section{Results}

\section{Discussion}


\begin{thebibliography}{70}


\end{thebibliography}


\end{document}
